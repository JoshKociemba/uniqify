\documentclass[letterpaper,10pt,titlepage]{article}

\usepackage{geometry}
\geometry{textheight=10in, textwidth=7.5in}

\usepackage{hyperref}

\def\name{Joshua Kociemba}

\hypersetup{
	pdfauthor = {\name},
	pdfkeywords = {cs311 ``operating systems'' files filesystem I/O},
	pdftitle = {CS 311 Project 2: UNIX Archiver},
	pdfsubject = {CS 311 Project 2},
	pdfpagemode = UseNone
}

\parindent = 0.0 in
\parskip = 0.2 in

\begin{document}

\begin{enumerate}
	\item A design for your system, as well as places your implementation
deviated from this design.
		\begin{enumerate}
			\item Start by designing the way archive files are
opened, if the archive doesn't exist handle the creation of the archive, and
when the program exits make sure the archive is closed. This functions will be
written first, because of their importance.
			\item Write proper functions for printing out the
content inside of the archive. All of the other functions will be useless if we
can't see the changes being made to the archive. I will start with the verbose
method of printing the archive, and then the concise option will be a walk in
the park. 
		\end{enumerate}

	\item A work log, detailing what you did when -- this can fairly easily
be created if you are using some form of revision control.
		\begin{enumerate}
			\item First Commit
			\item Added myar.c
			\item Added makefile
			\item Added myar.h
			\item Modified myar.c - updated main function to handle
getopt arguments - still unfinished
			\item Modified myar.c - missing break statements
			\item Wrote arch\_open and arch\_close
			\item Started to write arch\_write\_ghdr
			\item Added writeup.tex
			\item Started verbose and concise prints
			\item Finished verbose and concise
			\item Re-wrote large portions of myar.c
			\item Added .tex compiler section to makefile
		\end{enumerate}		

	\item Any challenges you overcame in completing this assignment answers
to the following questions:
		\begin{enumerate}
			\item What do you think the main point of this
assignment is? Answer: I think the main point of this assignment was to learn
more about system calls and file I/O. I learned a lot about how lseek, read, and
write works in this assignment. I have the format of an arch file almost
engraved into my memory.
			\item How did you ensure your solution was correct?
Testing details, for instance. Answer: My main testing method was comparison
between my program and UNIX ar. Seeing how ar handles the different options and
trying to recreate that as much as possible. I had a lot of trial and error when
it came to reading from the archive file. I had problems with over-reading the
header, or under-reading it. It was a tough problem to try and fix, because I
had a hard time figuring out whether I was over or under reading the header
line. I had other problems with figuring out how to get started with the
assignment. But after getting a good design thought out for the system, things
started to click.
			\item What did you learn? Answer: The main thing I
learned from this project was to have a solid design before I even start
programming. Having small sections to get working is a lot more manageable than
trying to do a very large portion of the assignment. I ended up running out of
time on this assignment because of that. I also learned alot about what archive
headers look like, and what each part of it means. I learned a lot about lseek(),
and even more about read().
		\end{enumerate}

\end{enumerate}

\end{document}
